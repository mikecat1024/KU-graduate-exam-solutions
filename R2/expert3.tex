\documentclass[dvipdfmx]{jsarticle}

\input{../preamble}

\begin{document}
    最初に$A[1/x] \cong A_x$であることを示す.
    \[
        \phi: A[1/x] \longrightarrow A_x
    \]
    を自然な全射準同型とする.
    $a/x^n, b/x^m \in A[1/x]$を$n \leq m$となるように任意にとったとき,
    \[
        \frac{a}{x^n} = \phi\qty(\frac{a}{x^n}) = \phi\qty(\frac{b}{x^m}) = \frac{b}{x^m}
    \]
    とすれば, ある$k$が存在して, $A$において, $x^k(x^{m-n}a-b) = 0$が成り立つ.
    ここで, $A$が整域であることと$x \neq 0$より, $x^{m-n}a = b$であり,
    $A[1/x]$において,
    \[
        \frac{b}{x^m} = \frac{x^{m-n}a}{x^m} = \frac{a}{x^n}
    \]
    となるので, $\phi$は単射である.
    したがって, $A[1/x] \cong A_x$が成り立つ.

    $A$が高さ$1$の素イデアルを持たない場合は主張が成り立つので, 高さ$1$の素イデアルは存在するとしてよい.
    $\mathfrak{p} \in \Spec{A}$を高さ$1$の素イデアルとする.
    また, $x \in \mathfrak{p}$のときには$\mathfrak{p}$の極小性と$xA \subseteq \mathfrak{p}$が素イデアルであることから,
    $\mathfrak{p} = xA$となるので, $x \notin \mathfrak{p}$としてよい.

    このとき, 局所化による素イデアルの対応関係から, $\mathfrak{p}A_x$も高さ$1$の素イデアルである.
    したがって, ある$a \in A_x$が存在して, $\mathfrak{p}A_x = aA_x$となる.
    特に$a \in \mathfrak{p}A_x$なので, ある$p \in \mathfrak{p}$と$n$が存在して, $a = p/x^n$が成り立つ.
    これより, $p = ax^n \in \mathfrak{p}$となり, $x \notin \mathfrak{p}$と$\mathfrak{p}$が素イデアルであることから,
    $a \in \mathfrak{p}$が成り立つ.
    ゆえに, $aA \subseteq \mathfrak{p}$である.

    次に, $y \in \mathfrak{p}$を任意にとる.
    このとき, ある$n$と$b \in A$が存在して, $x^ny = ab$となる.
    ここで,
    \[
        I_n = \sum_{x^nz \in aA} zA
    \]
    とおく.
    すると, $x^nz \in aA \subseteq \mathfrak{p}$ならば$z \in \mathfrak{p}$であることから,
    $I_n \subseteq \mathfrak{p}$であり,
    \[
        I_0 \subseteq I_1 \subseteq \cdots \subseteq I_n \subseteq \cdots \subseteq \mathfrak{p}
    \]
    となる.
    ここで, $A$はNoetherianなので, ある$m$が存在して, 任意の$k \geq m$に対して, $I_m = I_k$が成り立つ.
    したがって, 任意の$y \in \mathfrak{p}$に対して, ある$b \in A$が存在して,
    $ab = x^my \in xA$が成り立つ.

    最後に, $\mathfrak{p}$が単項イデアルであることを示す.
    $a \notin xA$のとき, $x^my = ab \in xA$と$x$が素元であることから, $b \in xA$となる.
    ゆえに, ある$b' \in A$が存在して, $b = b'x$が成り立つ.
    このとき, $x^my = ab'x$と$A$が整域であることから, $x^{m-1}y  = ab'$となるので, これを繰り返すことにより,
    $y \in aA$が従う.

    あとは$a \in xA$の場合を考えればよい.
    $a \in x^{k_0}A$を満たす最大, または$m$以上の$k_0$をとり, $k = \min(k_0, m)$と定める.
    このとき, ある$a' \in A$が存在して, $a = a'x^k$となるので, 上と同様にして,
    $x^{m-k}y = a'b$が成り立つ.
    $m = k$ならば$y \in a'A$であり,
    $m - k \geq 1$の場合には, $a' \notin xA$なので, $a \notin xA$の場合と同様にして,
    $y \in a'A$が従う.
    これらより, $\mathfrak{p} \subseteq a'A$となる.
    しかし, $a'x^k = a \in \mathfrak{p}$なので, $x \notin \mathfrak{p}$より, $a' \in \mathfrak{p}$となって,
    $\mathfrak{p} = a'A$が成り立つ.
\end{document}
