\documentclass[dvipdfmx]{jsarticle}

\usepackage{amsthm}
\usepackage{enumerate}
\usepackage{amsmath}
\usepackage{amssymb}
\usepackage{cleveref}
\usepackage{physics}
\usepackage{color}
\usepackage{tikz-cd}

\newcounter{BaseCounter}[section]

\newtheoremstyle{JapanesePropositionStyle}{24pt}{}{}{}{\bfseries}{.}{1em}{\thmname{#1}\hspace{0.2em}\thmnumber{#2}\thmnote{\hspace{0.5em}(#3)}}
\theoremstyle{JapanesePropositionStyle}

\newtheorem{proposition}[BaseCounter]{Proposition}
\newtheorem{note}[BaseCounter]{Note}
\newtheorem{lemma}[BaseCounter]{Lemma}
\newtheorem{corollary}[BaseCounter]{Corollary}
\newtheorem{theorem}[BaseCounter]{Theorem}
\newtheorem{definition}[BaseCounter]{Definition}
\newtheorem{example}[BaseCounter]{Example}

\makeatletter
\newcommand\RedeclareMathOperator{%
  \@ifstar{\def\rmo@s{m}\rmo@redeclare}{\def\rmo@s{o}\rmo@redeclare}%
}
\newcommand\rmo@redeclare[2]{%
  \begingroup \escapechar\m@ne\xdef\@gtempa{{\string#1}}\endgroup
  \expandafter\@ifundefined\@gtempa
     {\@latex@error{\noexpand#1undefined}\@ehc}%
     \relax
  \expandafter\rmo@declmathop\rmo@s{#1}{#2}}
\newcommand\rmo@declmathop[3]{%
  \DeclareRobustCommand{#2}{\qopname\newmcodes@#1{#3}}%
}
\@onlypreamble\RedeclareMathOperator
\makeatother

\DeclareMathOperator{\ch}{char}
\DeclareMathOperator{\Hom}{Hom}
\DeclareMathOperator{\Emb}{Emb}
\DeclareMathOperator{\Aut}{Aut}
\DeclareMathOperator{\sign}{sign}
\DeclareMathOperator{\supp}{supp}
\DeclareMathOperator{\interior}{int}
\DeclareMathOperator{\relint}{relint}
\DeclareMathOperator{\Div}{Div}
\DeclareMathOperator{\Spec}{Spec}
\DeclareMathOperator{\coker}{coker}

\DeclareMathOperator*{\colim}{colim}

\RedeclareMathOperator{\Im}{Im}

\newcommand{\emb}{\hookrightarrow}

\newcommand{\Mod}[1]{\textbf{Mod}_{#1}}
\newcommand{\fMod}[1]{\textbf{fMod}_{#1}}
\newcommand{\QCoh}[1]{\textbf{QCoh}_{#1}}
\newcommand{\Coh}[1]{\textbf{Coh}_{#1}}

\renewcommand{\theenumi}{\arabic{enumi}}
\renewcommand{\labelenumi}{(\theenumi)}


\begin{document}
  \begin{enumerate}
    \item 有限アーベル群の構造定理より, $a_1, a_2, \dots, a_n$であって,
    \[
      G \cong \prod_i \mathbb{Z}/a_i\mathbb{Z} =: \prod_i H_{i}
    \]
    かつ$a_{i}$は$a_{i+1}$の約数であるようなものが存在する.
    このとき, $n \geq 2$と仮定すれば, $a_{n-1}$は$a_n$の約数であって, $H_{n}$は巡回群なので,
    位数$a_{n-1}$の巡回群を部分群として含む.
    しかし, $G$は条件$(\ast)$を満たすので, $H_{n-1} \subseteq H_{n}$が成り立つ.
    これは矛盾なので, $n = 1$となる.
    ゆえに, $G$は巡回群である.
    \item $G/H$の部分群の集合と$G$の$H$を含む部分群の集合
    には包含関係を保存するような自然な全単射が存在する.
    ゆえに, ある$k$が存在して, 位数$k$の部分群$N_1/H, N_2/H \subseteq G/H$が存在したと仮定すれば,
    \[
      |N_1|/|H| = |N_1/H| = k = |N_2/H| = |N_2|/|H|
    \]
    より, $|N_1| = |N_2|$となって, $G$が条件$(\ast)$を満たすことに反する.
    したがって, $G/H$も条件$(\ast)$を満たす.
    \item $n = |G|$として, $n = 1$の場合には主張は明らかに成り立つ.

    $n \geq 2$のとき, $n-1$以下の位数を持つ有限群で, 条件$(\ast)$を満たすものは
    巡回群であると仮定する.
    特に, $G$の真部分群はすべて巡回群であると仮定する.
    $|G| = p_1^{e_1}\cdots p_m^{e_m}$とおき, $G$のSylow-$p_k$部分群を$H_k$とする.
    $m \geq 2$のとき, $H_k$は帰納法の仮定より巡回群であり, その生成元を$h_k$とおく.
    また, 条件$(\ast)$より, $G$の任意の部分集合について, 共役な部分群は自分自身のみなので, 正規部分群である.
    これより, $i \neq j$のとき, $[H_i, H_j] \subseteq H_i \cap H_j = 1$となるので, $H_i$と$H_j$の元はそれぞれ可換である.
    さらに, $\gcd(p_i, p_j) = 1$より, $h_ih_j$で生成される巡回群$H_iH_j$の位数は$p_ip_j$なので,
    \[
      H_i \times H_j \ni (h_i^a, h_j^b) \longmapsto h_i^ah_j^b \in H_iH_j
    \]
    は全単射準同型である.
    ゆえに, 中国式剰余定理より,
    \[
      H_iH_j \cong H_i \times H_j \cong \mathbb{Z}/p_i^{e_i}\mathbb{Z} \times \mathbb{Z}/p_j^{e_j}\mathbb{Z} \cong \mathbb{Z}/p_i^{e_i}p_j^{e_j}\mathbb{Z}
    \]
    となる.
    これは巡回群かつ$G$の正規部分群なので, これを他のSylow部分群に対しても繰り返すことにより,
    $G$が巡回群であることが従う.
  \end{enumerate}
\end{document}
