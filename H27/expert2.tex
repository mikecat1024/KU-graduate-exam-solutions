
\documentclass[dvipdfmx]{jsarticle}

\usepackage{amsthm}
\usepackage{enumerate}
\usepackage{amsmath}
\usepackage{amssymb}
\usepackage{cleveref}
\usepackage{physics}
\usepackage{color}
\usepackage{tikz-cd}

\newcounter{BaseCounter}[section]

\newtheoremstyle{JapanesePropositionStyle}{24pt}{}{}{}{\bfseries}{.}{1em}{\thmname{#1}\hspace{0.2em}\thmnumber{#2}\thmnote{\hspace{0.5em}(#3)}}
\theoremstyle{JapanesePropositionStyle}

\newtheorem{proposition}[BaseCounter]{Proposition}
\newtheorem{note}[BaseCounter]{Note}
\newtheorem{lemma}[BaseCounter]{Lemma}
\newtheorem{corollary}[BaseCounter]{Corollary}
\newtheorem{theorem}[BaseCounter]{Theorem}
\newtheorem{definition}[BaseCounter]{Definition}
\newtheorem{example}[BaseCounter]{Example}

\makeatletter
\newcommand\RedeclareMathOperator{%
  \@ifstar{\def\rmo@s{m}\rmo@redeclare}{\def\rmo@s{o}\rmo@redeclare}%
}
\newcommand\rmo@redeclare[2]{%
  \begingroup \escapechar\m@ne\xdef\@gtempa{{\string#1}}\endgroup
  \expandafter\@ifundefined\@gtempa
     {\@latex@error{\noexpand#1undefined}\@ehc}%
     \relax
  \expandafter\rmo@declmathop\rmo@s{#1}{#2}}
\newcommand\rmo@declmathop[3]{%
  \DeclareRobustCommand{#2}{\qopname\newmcodes@#1{#3}}%
}
\@onlypreamble\RedeclareMathOperator
\makeatother

\DeclareMathOperator{\ch}{char}
\DeclareMathOperator{\Hom}{Hom}
\DeclareMathOperator{\Emb}{Emb}
\DeclareMathOperator{\Aut}{Aut}
\DeclareMathOperator{\sign}{sign}
\DeclareMathOperator{\supp}{supp}
\DeclareMathOperator{\interior}{int}
\DeclareMathOperator{\relint}{relint}
\DeclareMathOperator{\Div}{Div}
\DeclareMathOperator{\Spec}{Spec}
\DeclareMathOperator{\coker}{coker}

\DeclareMathOperator*{\colim}{colim}

\RedeclareMathOperator{\Im}{Im}

\newcommand{\emb}{\hookrightarrow}

\newcommand{\Mod}[1]{\textbf{Mod}_{#1}}
\newcommand{\fMod}[1]{\textbf{fMod}_{#1}}
\newcommand{\QCoh}[1]{\textbf{QCoh}_{#1}}
\newcommand{\Coh}[1]{\textbf{Coh}_{#1}}

\renewcommand{\theenumi}{\arabic{enumi}}
\renewcommand{\labelenumi}{(\theenumi)}


\begin{document}
    \begin{enumerate}
        \item $\phi(x) = x'$, $\phi(y) = x't$なので,
        $\phi(y^2-6x^2) = X^2(T^2-6) \in (X^2-6)$となり, well-defined.

        $f \in \mathbb{Z}[X,Y]/(Y^2-6x^2) \setminus \set{0}$の代表元を$f_0 \in \mathbb{Z}[X,T]$とする.
        このとき, $\deg_Y{f_0} \leq 1$と仮定してよいので,
        ある$g, h \in \mathbb{Z}[X]$が存在して, $f_0 = gY+ h$と表せる.
        このとき,
        $\phi(f_0) = gXT + h$であって,
        $gXT + h \in (T^2-6)$と仮定すれば, $\mathbb{Z}[X,T]$は整域であって,
        $1, T$は$\mathbb{Z}[X]$上線形独立なので,
        $gX = h = 0$となる.
        さらに, $\mathbb{Z}[X]$も整域なので, $g = h = 0$となるが,
        $f_0 \neq 0$に反するので, $\phi$は単射である.
        \item $A/P_1$が整域であることを示す.
        まず, $A = \mathbb{Z}[X,Y]/(Y^2-6X^2)$のイデアルは$(Y^2-6X^2)$を含む
        $\mathbb{Z}[X,Y]$のイデアルと対応しており,
        ある$I_1 \subseteq \mathbb{Z}[X,Y]$が存在して,
        $P_1 = I_1/(Y^2-6X^2)$と表せる.
        このとき, $(X,Y,5) \subseteq I_1$なので,
        $(X,Y,5) + (Y^2-6X^2) \subseteq I_1$となる.
        ここで,
        \[
            P_1 \subseteq ((X,Y,5) + (Y^2-6X^2))/(Y^2-6X^2) \subseteq I_1/(Y^2-6X^2) = P_1
        \]
        なので,
        \[
            P_1 = ((X,Y,5) + (Y^2-6X^2))/(Y^2-6X^2)
        \]
        であり,
        \[
            (X,Y,5) + (Y^2-6X^2) = (X, Y, 5)
        \]
        も成り立つので,
        \[
            A/P_1 = (\mathbb{Z}[X, Y]/(Y^2-6X^2))/((X,Y,5)/(Y^2-6X^2)) = \mathbb{Z}[X, Y]/(X,Y,5) = \mathbb{Z}/5\mathbb{Z}
        \]
        となる.
        これは整域なので, $P_1$は素イデアルである.

        次に, $A/P_2$が整域であることを示す.
        $A/P_1$の場合と同様にして,
        \[
            P_2 = ((X-Y, 5) + (Y^2-6X^2))/(Y^2-6X^2)
        \]
        であり,
        \[
            (X-Y, 5) + (Y^2-6X^2) = (X-Y, 5)
        \]
        なので,
        \[
            A/P_2 = (\mathbb{Z}[X, Y]/(Y^2-6X^2))/((X-Y,5)/(Y^2-6X^2)) = \mathbb{Z}[X, Y]/(X-Y,5) = \mathbb{Z}/5\mathbb{Z}[X]
        \]
        となる.
        これは整域なので, $P_2$は素イデアルである.

        $A/P_1 \neq A/P_2$より, $P_1 \neq P_2$であって,
        $P_2 \subseteq P_1$は明らかに成り立つので,
        $P_2 \subsetneq P_1$となる.
        \item $B/Q_1$が整域であることを示す.
        (2)と同様にして,
        \[
            B/Q_1 = (\mathbb{Z}[X, T]/(T^2-6))/((X, T+1) + (T^2 -6)/(T^2-6))
        \]
        であって,
        \[
            (X, T+1) + (T^2 - 6) = (X, T+1, 5)
        \]
        なので,
        \[
            B/Q_1 = \mathbb{Z}[X, T]/(X, T+1, 5)  = \mathbb{Z}/5\mathbb{Z}
        \]
        となる.
        これは整域なので, $Q_1$は素イデアルである.

        また, (2)で示したことから, $P_1$は特に極大イデアルであって, $Q_1 \neq 1$なので,
        $P_1 \subseteq Q_1 \cap A$を示せば十分である.
        $\phi(X) \in Q_1$は明らかであり,
        \[
            \phi(Y) = TX = X(T+1)-X \in Q_1
        \]
        と,
        \[
            \phi(5) = 5 = -(T^2-6) + (T-1)(T+1) \in Q_1
        \]
        も成り立つので, $P_1 \subseteq Q_1 \cap A$が成り立つ.
        これより, $P_1 = Q_1 \cap A$となる.
        \item $\mathbb{Z}$がネーターであることから, $\mathbb{Z}[X, T] = 2$であり,
        $\mathbb{Z}[X, T]$において
        \[
            0 \subsetneq (T^2-6) \subsetneq (X, T+1) = Q_1
        \]
        が成り立つこととイデアルの対応関係から,
        $B$の素イデアルであって, $Q_1$に真に含まれる
        素イデアルは$0$しか存在しないが, $P_2 \neq 0$なので,
        条件を満たす$Q_2$は存在しない.
    \end{enumerate}
\end{document}
