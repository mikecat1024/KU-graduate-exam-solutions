
\documentclass[dvipdfmx]{jsarticle}

\usepackage{amsthm}
\usepackage{enumerate}
\usepackage{amsmath}
\usepackage{amssymb}
\usepackage{cleveref}
\usepackage{physics}
\usepackage{color}
\usepackage{tikz-cd}

\newcounter{BaseCounter}[section]

\newtheoremstyle{JapanesePropositionStyle}{24pt}{}{}{}{\bfseries}{.}{1em}{\thmname{#1}\hspace{0.2em}\thmnumber{#2}\thmnote{\hspace{0.5em}(#3)}}
\theoremstyle{JapanesePropositionStyle}

\newtheorem{proposition}[BaseCounter]{Proposition}
\newtheorem{note}[BaseCounter]{Note}
\newtheorem{lemma}[BaseCounter]{Lemma}
\newtheorem{corollary}[BaseCounter]{Corollary}
\newtheorem{theorem}[BaseCounter]{Theorem}
\newtheorem{definition}[BaseCounter]{Definition}
\newtheorem{example}[BaseCounter]{Example}

\makeatletter
\newcommand\RedeclareMathOperator{%
  \@ifstar{\def\rmo@s{m}\rmo@redeclare}{\def\rmo@s{o}\rmo@redeclare}%
}
\newcommand\rmo@redeclare[2]{%
  \begingroup \escapechar\m@ne\xdef\@gtempa{{\string#1}}\endgroup
  \expandafter\@ifundefined\@gtempa
     {\@latex@error{\noexpand#1undefined}\@ehc}%
     \relax
  \expandafter\rmo@declmathop\rmo@s{#1}{#2}}
\newcommand\rmo@declmathop[3]{%
  \DeclareRobustCommand{#2}{\qopname\newmcodes@#1{#3}}%
}
\@onlypreamble\RedeclareMathOperator
\makeatother

\DeclareMathOperator{\ch}{char}
\DeclareMathOperator{\Hom}{Hom}
\DeclareMathOperator{\Emb}{Emb}
\DeclareMathOperator{\Aut}{Aut}
\DeclareMathOperator{\sign}{sign}
\DeclareMathOperator{\supp}{supp}
\DeclareMathOperator{\interior}{int}
\DeclareMathOperator{\relint}{relint}
\DeclareMathOperator{\Div}{Div}
\DeclareMathOperator{\Spec}{Spec}
\DeclareMathOperator{\coker}{coker}

\DeclareMathOperator*{\colim}{colim}

\RedeclareMathOperator{\Im}{Im}

\newcommand{\emb}{\hookrightarrow}

\newcommand{\Mod}[1]{\textbf{Mod}_{#1}}
\newcommand{\fMod}[1]{\textbf{fMod}_{#1}}
\newcommand{\QCoh}[1]{\textbf{QCoh}_{#1}}
\newcommand{\Coh}[1]{\textbf{Coh}_{#1}}

\renewcommand{\theenumi}{\arabic{enumi}}
\renewcommand{\labelenumi}{(\theenumi)}


\begin{document}
    $R_a = K[X, Y]/ (X^2-Y^2-X-Y-a) \cong K[S,T]/(ST-a)$であることに注意する.
    $I = (ST-a)$とおく.
    \begin{enumerate}
        \item Hilbertの零点定理より, $K[X, Y]$
        の極大イデアルは$(X-\alpha, Y-\beta)$という形をしている.
        また, 剰余環のイデアルの対応関係により,
        $R_a$の極大イデアルは$\mathfrak{m} = (X-\alpha, Y-\beta)/I$と表せる.
        ここで, $(XY) = I \subseteq (X-\alpha, Y-\beta)$より,
        $\alpha\beta = 0$となるので, $\alpha = 0$または$\beta = 0$が成り立つ.
        $\beta = 0$としても一般性は失わないので, 以下では$\beta = 0$を仮定する.
        \[
            ((X-\alpha, Y)/I)^2 = ((X-\alpha, Y)^2 + I)/I
        \]
        であることに注意し, $\mathfrak{m}_\alpha = (X-\alpha, Y)$とおけば,
        \[
            (\mathfrak{m}_\alpha/I)/(\mathfrak{m}_\alpha/I)^2 = (\mathfrak{m}_\alpha/I)/((\mathfrak{m}_\alpha^2+I)/I) = \mathfrak{m}_\alpha/(\mathfrak{m}_\alpha^2 + I)
        \]
        となる.

        $\alpha = 0$とすれば,
        \[
            \mathfrak{m}_0/(\mathfrak{m}_0^2 + I) = (X,Y)/((X^2, XY, Y^2) + (XY)) = (X,Y)/(X^2, XY, Y^2)
        \]
        となるので, $\dim_K(\mathfrak{m}/\mathfrak{m}^2) = 2$となる.
        これと次元理論の基本定理より, $\mathfrak{m}$が
        単項生成でないことが従う.

        $\alpha \neq 0$とすれば,
    \end{enumerate}
\end{document}
