
\documentclass[dvipdfmx]{jsarticle}

\usepackage{amsthm}
\usepackage{enumerate}
\usepackage{amsmath}
\usepackage{amssymb}
\usepackage{cleveref}
\usepackage{physics}
\usepackage{color}
\usepackage{tikz-cd}

\newcounter{BaseCounter}[section]

\newtheoremstyle{JapanesePropositionStyle}{24pt}{}{}{}{\bfseries}{.}{1em}{\thmname{#1}\hspace{0.2em}\thmnumber{#2}\thmnote{\hspace{0.5em}(#3)}}
\theoremstyle{JapanesePropositionStyle}

\newtheorem{proposition}[BaseCounter]{Proposition}
\newtheorem{note}[BaseCounter]{Note}
\newtheorem{lemma}[BaseCounter]{Lemma}
\newtheorem{corollary}[BaseCounter]{Corollary}
\newtheorem{theorem}[BaseCounter]{Theorem}
\newtheorem{definition}[BaseCounter]{Definition}
\newtheorem{example}[BaseCounter]{Example}

\makeatletter
\newcommand\RedeclareMathOperator{%
  \@ifstar{\def\rmo@s{m}\rmo@redeclare}{\def\rmo@s{o}\rmo@redeclare}%
}
\newcommand\rmo@redeclare[2]{%
  \begingroup \escapechar\m@ne\xdef\@gtempa{{\string#1}}\endgroup
  \expandafter\@ifundefined\@gtempa
     {\@latex@error{\noexpand#1undefined}\@ehc}%
     \relax
  \expandafter\rmo@declmathop\rmo@s{#1}{#2}}
\newcommand\rmo@declmathop[3]{%
  \DeclareRobustCommand{#2}{\qopname\newmcodes@#1{#3}}%
}
\@onlypreamble\RedeclareMathOperator
\makeatother

\DeclareMathOperator{\ch}{char}
\DeclareMathOperator{\Hom}{Hom}
\DeclareMathOperator{\Emb}{Emb}
\DeclareMathOperator{\Aut}{Aut}
\DeclareMathOperator{\sign}{sign}
\DeclareMathOperator{\supp}{supp}
\DeclareMathOperator{\interior}{int}
\DeclareMathOperator{\relint}{relint}
\DeclareMathOperator{\Div}{Div}
\DeclareMathOperator{\Spec}{Spec}
\DeclareMathOperator{\coker}{coker}

\DeclareMathOperator*{\colim}{colim}

\RedeclareMathOperator{\Im}{Im}

\newcommand{\emb}{\hookrightarrow}

\newcommand{\Mod}[1]{\textbf{Mod}_{#1}}
\newcommand{\fMod}[1]{\textbf{fMod}_{#1}}
\newcommand{\QCoh}[1]{\textbf{QCoh}_{#1}}
\newcommand{\Coh}[1]{\textbf{Coh}_{#1}}

\renewcommand{\theenumi}{\arabic{enumi}}
\renewcommand{\labelenumi}{(\theenumi)}


\begin{document}
    $R_a = K[X, Y]/ (X^2-Y^2-X-Y-a) \cong K[S,T]/(ST-a)$なので,
    $R_a = K[X, Y]/(XY-a)$として考えればよい.
    \begin{enumerate}
        \item Hilbertの零点定理より,
        $K[X, Y]$の極大イデアルは$\alpha, \beta \in K$を使って,
        $(X-\alpha, Y-\beta)$と表せる.
        これを$\mathfrak{m}_{\alpha, \beta}$と定める.

        剰余環のイデアルの対応関係より,
        $\mathfrak{m}$を$R_0$の極大イデアルとすれば,
        $\mathfrak{m} = \mathfrak{m}_{\alpha, \beta} /(XY)$と表せる.
        このとき, $(XY) \subseteq \mathfrak{m}_{\alpha, \beta}$より,
        $X = \alpha, Y = \beta$を代入する写像を考えれば,
        $\alpha\beta = 0$となる.
        また, $K$は特に整域なので,
        $\alpha = 0$または$\beta = 0$である.
        以下, $\beta = 0$と仮定する.
        $\alpha = 0$のとき,
        \[
            \mathfrak{m}/\mathfrak{m}^2 = (\mathfrak{m}_{0,0}/(XY)) /(\mathfrak{m}_{0,0}/(XY))^2 = \mathfrak{m}_{0,0}/(\mathfrak{m}_{0,0}^2 + (XY)) = (X,Y)/(X^2, Y^2, XY)
        \]
        となるので,
        $\dim_{K} \mathfrak{m}/\mathfrak{m}^2 = 2$である.
        また,
        $\alpha \neq 0$のときには,
        \[
            \mathfrak{m}/\mathfrak{m}^2 = (\mathfrak{m}_{\alpha,0}/(XY)) /(\mathfrak{m}_{\alpha,0}/(XY))^2 = \mathfrak{m}_{\alpha,0}/(\mathfrak{m}_{\alpha,0}^2 + (XY)) = (X-\alpha, Y)/((X-\alpha)^2, Y)
        \]
        となるので,
        $\dim_K \mathfrak{m}/\mathfrak{m}^2 = 1$となる.

        さらに, $\mathfrak{m}$が単項イデアルの場合は
        \[
            \mathfrak{m}/\mathfrak{m}^2 = \mathfrak{m} \otimes_{R_0} R_0/\mathfrak{m} = R_0/\mathfrak{m} = K
        \]
        となるので,
        $\dim_K \mathfrak{m}/\mathfrak{m}^2 = 1$となる.
        したがって,
        $\alpha = 0$の場合には$\mathfrak{m}$は単項イデアルでない.
        また, $\alpha \neq 0$の場合には$\mathfrak{m} = (X-\alpha)R_0$となるので, $\mathfrak{m}$は単項イデアルになる.
        \item (1)と同様にして, $\alpha\beta = a \neq 0$であって,
        $K$は特に整域なので, $\alpha \neq 0$かつ$\beta \neq 0$が成り立つ.
        このとき, $\mathfrak{m} = (X-\alpha)R_a$なので, 単項イデアルであって, 
        (1)で示していることから,
        $\dim_K \mathfrak{m}/\mathfrak{m}^2 = 1$が成り立つ.
    \end{enumerate}
\end{document}
