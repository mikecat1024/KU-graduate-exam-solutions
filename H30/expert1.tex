\documentclass[dvipdfmx]{jsarticle}

\input{../preamble}

\begin{document}
    \begin{description}
        \item[(ii) $\Rightarrow$ (i)]
            $k$を代数閉体とし, ある$f \in k[X,Y]$が存在して,
            $V(f) = \set{(0,0)}$と仮定する.
            このとき,
            \[
                V(f) = \set{(0,0)} = V((X,Y))
            \]
            なので, Hilbertの零点定理より, $\sqrt{(f)} = \sqrt{(X,Y)} = (X,Y)$が成り立つ.
            ここで, $X, Y \in (X,Y)$なので, ある$n, m$が存在して,
            $X^n ,Y^m \in (f)$となる.
            ゆえに, ある$g, h \in k[X, Y]$が存在して,
            $X^n = fg$かつ$Y^m = fh$が成り立つ.
            したがって, $f$が定数または$n  = m = 0$となる.
            $f$が定数の場合は明らかに矛盾であり,
            $n = m = 0$の場合は, $g = h = f^{-1}$なので,
            $V(f) \neq \emptyset$であることに矛盾.
            これらより, 任意の$f \in k[X, Y]$について, $V(f) \neq \set{(0,0)}$が成り立つ.
        \item[(i) $\Rightarrow$ (ii)]
            $k$は代数閉ではないので, 二次以上の既約多項式$g \in k[X]$が存在して, $g$は$k$上で根をもたない.
            この$g$に対して, $f = Y^{\deg{g}}g(X/Y)$が(ii)を満たす.
    \end{description}
\end{document}
