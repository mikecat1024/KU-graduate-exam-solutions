\documentclass[dvipdfmx]{jsarticle}

\usepackage{amsthm}
\usepackage{enumerate}
\usepackage{amsmath}
\usepackage{amssymb}
\usepackage{cleveref}
\usepackage{physics}
\usepackage{color}
\usepackage{tikz-cd}

\newcounter{BaseCounter}[section]

\newtheoremstyle{JapanesePropositionStyle}{24pt}{}{}{}{\bfseries}{.}{1em}{\thmname{#1}\hspace{0.2em}\thmnumber{#2}\thmnote{\hspace{0.5em}(#3)}}
\theoremstyle{JapanesePropositionStyle}

\newtheorem{proposition}[BaseCounter]{Proposition}
\newtheorem{note}[BaseCounter]{Note}
\newtheorem{lemma}[BaseCounter]{Lemma}
\newtheorem{corollary}[BaseCounter]{Corollary}
\newtheorem{theorem}[BaseCounter]{Theorem}
\newtheorem{definition}[BaseCounter]{Definition}
\newtheorem{example}[BaseCounter]{Example}

\makeatletter
\newcommand\RedeclareMathOperator{%
  \@ifstar{\def\rmo@s{m}\rmo@redeclare}{\def\rmo@s{o}\rmo@redeclare}%
}
\newcommand\rmo@redeclare[2]{%
  \begingroup \escapechar\m@ne\xdef\@gtempa{{\string#1}}\endgroup
  \expandafter\@ifundefined\@gtempa
     {\@latex@error{\noexpand#1undefined}\@ehc}%
     \relax
  \expandafter\rmo@declmathop\rmo@s{#1}{#2}}
\newcommand\rmo@declmathop[3]{%
  \DeclareRobustCommand{#2}{\qopname\newmcodes@#1{#3}}%
}
\@onlypreamble\RedeclareMathOperator
\makeatother

\DeclareMathOperator{\ch}{char}
\DeclareMathOperator{\Hom}{Hom}
\DeclareMathOperator{\Emb}{Emb}
\DeclareMathOperator{\Aut}{Aut}
\DeclareMathOperator{\sign}{sign}
\DeclareMathOperator{\supp}{supp}
\DeclareMathOperator{\interior}{int}
\DeclareMathOperator{\relint}{relint}
\DeclareMathOperator{\Div}{Div}
\DeclareMathOperator{\Spec}{Spec}
\DeclareMathOperator{\coker}{coker}

\DeclareMathOperator*{\colim}{colim}

\RedeclareMathOperator{\Im}{Im}

\newcommand{\emb}{\hookrightarrow}

\newcommand{\Mod}[1]{\textbf{Mod}_{#1}}
\newcommand{\fMod}[1]{\textbf{fMod}_{#1}}
\newcommand{\QCoh}[1]{\textbf{QCoh}_{#1}}
\newcommand{\Coh}[1]{\textbf{Coh}_{#1}}

\renewcommand{\theenumi}{\arabic{enumi}}
\renewcommand{\labelenumi}{(\theenumi)}


\begin{document}
    \begin{description}
        \item[(ii) $\Rightarrow$ (i)]
            $k$を代数閉体とし, ある$f \in k[X,Y]$が存在して,
            $V(f) = \set{(0,0)}$と仮定する.
            このとき,
            \[
                V(f) = \set{(0,0)} = V((X,Y))
            \]
            なので, Hilbertの零点定理より, $\sqrt{(f)} = \sqrt{(X,Y)} = (X,Y)$が成り立つ.
            ここで, $X, Y \in (X,Y)$なので, ある$n, m$が存在して,
            $X^n ,Y^m \in (f)$となる.
            ゆえに, ある$g, h \in k[X, Y]$が存在して,
            $X^n = fg$かつ$Y^m = fh$が成り立つ.
            したがって, $f$が定数または$n  = m = 0$となる.
            $f$が定数の場合は明らかに矛盾であり,
            $n = m = 0$の場合は, $g = h = f^{-1}$なので,
            $V(f) \neq \emptyset$であることに矛盾.
            これらより, 任意の$f \in k[X, Y]$について, $V(f) \neq \set{(0,0)}$が成り立つ.
        \item[(i) $\Rightarrow$ (ii)]
            $k$は代数閉ではないので, 二次以上の既約多項式$g \in k[X]$が存在して, $g$は$k$上で根をもたない.
            この$g$に対して, $f = Y^{\deg{g}}g(X/Y)$が(ii)を満たす.
    \end{description}
\end{document}
